\documentclass{article}

\usepackage{booktabs}
\usepackage{tabularx}

\title{Development Plan\\\progname}

\author{\authname}

\date{}

%% Comments

\usepackage{color}

\newif\ifcomments\commentstrue %displays comments
%\newif\ifcomments\commentsfalse %so that comments do not display

\ifcomments
\newcommand{\authornote}[3]{\textcolor{#1}{[#3 ---#2]}}
\newcommand{\todo}[1]{\textcolor{red}{[TODO: #1]}}
\else
\newcommand{\authornote}[3]{}
\newcommand{\todo}[1]{}
\fi

\newcommand{\wss}[1]{\authornote{blue}{SS}{#1}} 
\newcommand{\plt}[1]{\authornote{magenta}{TPLT}{#1}} %For explanation of the template
\newcommand{\an}[1]{\authornote{cyan}{Author}{#1}}

%% Common Parts

\newcommand{\progname}{Software Engineering} % PUT YOUR PROGRAM NAME HERE
\newcommand{\authname}{Team 6, EcoOptimizers
\\ Nivetha Kuruparan
\\ Sevhena Walker
\\ Tanveer Brar
\\ Mya Hussain
\\ Ayushi Amin} % AUTHOR NAMES                  

\usepackage{hyperref}
    \hypersetup{colorlinks=true, linkcolor=blue, citecolor=blue, filecolor=blue,
                urlcolor=blue, unicode=false}
    \urlstyle{same}
                                


\begin{document}

\maketitle

\begin{table}[hp]
  \caption{Revision History} \label{TblRevisionHistory}
  \begin{tabularx}{\textwidth}{llX}
    \toprule
    \textbf{Date} & \textbf{Developer(s)} & \textbf{Change}\\
    \midrule
    September 18th, 2024 & All & Created first draft of document\\
    September 23rd, 2024 & All & Finalized document\\
    \bottomrule
  \end{tabularx}
\end{table}

\newpage{}

This document outlines the development plan for improving the energy efficiency of
engineered software through refactoring. It includes details on intellectual property,
team roles, workflow structure, and project scheduling. Additionally, the plan covers
expected technologies, coding standards, and proof of concept demonstrations, providing
a clear roadmap for the project's progression.

\section{Confidential Information?}

\wss{State whether your project has confidential information from industry, or
  not.  If there is confidential information, point to the agreement you have in
place.}

\wss{For most teams this section will just state that there is no confidential
information to protect.}
\section{IP to Protect}

\hspace{\parindent}The software and associated documentation files for this project are protected by copyright, outlined in the document \href{https://github.com/ssm-lab/capstone--source-code-optimizer/blob/documentation/LICENSE}{ here}(referred to as "License"). The License does not grant any party the rights to modify, merge, publish, distribute, sublicense, or sell the software without explicit permission .
Unauthorized use, modification, or distribution of the software is prohibited and may result in legal action.

Permission is granted on a case-by-case basis, non-transferable and non-exclusive. No rights to exploit the software commercially or otherwise are granted without prior written consent from the copyright holders.

\section{Copyright License}

\wss{What copyright license is your team adopting.  Point to the license in your
repo.}

\section{Team Meeting Plan}

The team will meet multiple times a week, once during Monday tutorial time and throughout the week as issues or concerns arise. The meetings will be conducted either online through a Teams meeting or in-person on campus. The team
will hold an official meeting with the industry advisor once a week yet the time has not been decided. The meeting with the advisor will be online on Teams for the first 3 weeks, followed by in-person meetings later on. Meetings itself will be strutured as follows:
\begin{enumerate}
  \item Each member will take turns giving a short recap of work they have accomplished throughout the week.
  \item Members will voice any concerns or issues they may be facing.
  \item Team will form a discussion and make decisions for the project.
  \item Any/all questions will be documented for the next meeting with the advisor.
\end{enumerate}
The team will go ahead and use Issues on Github to add anything they may want to talk about in the next meeting as the week progresses in order to have some form of agenda for the next meeting.

\section{Team Communication Plan}

\textbf{Issues}: GitHub \\
\textbf{Meetings}: Microsoft Teams \\
\textbf{Meetings (with advisor)}: Discord \\
\textbf{Informal Project Discussion}: WhatsApp

\section{Team Member Roles}

\hspace{\parindent}As a team, we will all function as developers, sharing responsibilities
for creating issues, coding, testing, and documentation in the early stages. Specific roles
will be defined as the project evolves, allowing for flexibility and collaboration.

During our scheduled meetings (with the supervisor, within the team, etc.), we will follow
an Agile Scrum structure, incorporating additional roles such as Scrum Master and Scribe.
These roles will rotate weekly, with the Scrum Master responsible for organizing and leading
the meetings, and the Scribe tasked with documenting key details. This approach ensures active
participation and shared responsibility amongst team members.

We have chosen not to designate a team leader, as we all possess similar skills and knowledge.
Instead, we aim to work collaboratively to resolve any challenges that arise.

\section{Workflow Plan}

The repository will contain two main persistent branches (branching off main) throughout the project:
\textbf{dev} and \textbf{documentation}. These branches, that we will henceforth call "epic" branches, will be used for technical
software development and documentation changes respectively. \\

The average workflow for the project will proceed as follows:
\begin{enumerate}
  \item Pull changes from the appropriate \textbf{epic} branch
  \item Create working branch from \textbf{epic} branch with the format <main contributor name>/<descriptive related to topic of changes>
  \item Create sub-working-branch from previous branch if necessary
  \item Commit changes with descriptive name
  \item Create \textbf{unit tests} for said changes
  \item Create pull request to merge changes from sub/working branch into working/epic branch
  \item Wait for all tests run with GitHub Actions to pass
  \item Wait for 2 approvals from teammates other than the one who created the Pull request
  \item Merge changes into target branch (working or epic)
\end{enumerate}

\noindent
GitHub Issues will be used to track and manage bugs, feature requests, and development tasks. Team members can report issues, propose enhancements, and assign tasks to specific individuals. Each issue will be labelled (e.g., bug, enhancement, team-meeting) for easy categorisation, linked to relevant milestones, and tracked through GitHub Project boards. Comments and code references will be exploited to allow for collaboration and discussion on potential solutions. Issues will also be integrated with pull requests, so they’ll automatically close once fixes are merged. \\

\noindent
For situations where a certain type of issue is projected to be created at a steady frequency, templates will be created to facilitate their creation. \\

\noindent
The CI/CD pipeline will be implemented via GitHub to improve automated testing as well as facilitate the feedback loop for the machine learning models used by the library. More detailed information will be added later. \\

\section{Project Decomposition and Scheduling}

\hspace{\parindent} The project is hosted on GitHub under the organization of Sustainable Systems and Methods (SSM) Lab and can be accessed at this \href{https://github.com/ssm-lab/capstone--source-code-optimizer}{link}.

The team currently has one GitHub Project setup to serve as a visual tool not only to track deadlines for all tasks but also for accountability  for assigned tasks. The project can include cards for:
\begin{itemize}
    \item Issues for course related deliverables
    \item Lecture and Meeting logs
    \item Issues for project specific tasks(such as building a certain component).
\end{itemize}

\wss{How will the project be scheduled?  This is the big picture schedule, not
  details. You will need to reproduce information that is in the course outline
for deadlines.}

\section{Proof of Concept Demonstration Plan}

What is the main risk, or risks, for the success of your project?  What will you
demonstrate during your proof of concept demonstration to convince yourself that
you will be able to overcome this risk?

\section{Expected Technology}

\wss{What programming language or languages do you expect to use?  What external
  libraries?  What frameworks?  What technologies.  Are there major components of
  the implementation that you expect you will implement, despite the existence of
  libraries that provide the required functionality.  For projects with machine
learning, will you use pre-trained models, or be training your own model?  }

\wss{The implementation decisions can, and likely will, change over the course
  of the project.  The initial documentation should be written in an abstract way;
  it should be agnostic of the implementation choices, unless the implementation
  choices are project constraints.  However, recording our initial thoughts on
  implementation helps understand the challenge level and feasibility of a
  project.  It may also help with early identification of areas where project
members will need to augment their training.}

Topics to discuss include the following:

\begin{itemize}
  \item Specific programming language
  \item Specific libraries
  \item Pre-trained models
  \item Specific linter tool (if appropriate)
  \item Specific unit testing framework
  \item Investigation of code coverage measuring tools
  \item Specific plans for Continuous Integration (CI), or an explanation that CI
    is not being done
  \item Specific performance measuring tools (like Valgrind), if
    appropriate
  \item Tools you will likely be using?
\end{itemize}

\wss{git, GitHub and GitHub projects should be part of your technology.}

\section{Coding Standard}

\wss{What coding standard will you adopt?}

\newpage{}

\section*{Appendix --- Reflection}

\wss{Not required for CAS 741}

The purpose of reflection questions is to give you a chance to assess your own
learning and that of your group as a whole, and to find ways to improve in the
future. Reflection is an important part of the learning process.  Reflection is
also an essential component of a successful software development process.  

Reflections are most interesting and useful when they're honest, even if the
stories they tell are imperfect. You will be marked based on your depth of
thought and analysis, and not based on the content of the reflections
themselves. Thus, for full marks we encourage you to answer openly and honestly
and to avoid simply writing ``what you think the evaluator wants to hear.''

Please answer the following questions.  Some questions can be answered on the
team level, but where appropriate, each team member should write their own
response:


\begin{enumerate}
  \item Why is it important to create a development plan prior to starting the
    project?
  \item In your opinion, what are the advantages and disadvantages of using
    CI/CD?
  \item What disagreements did your group have in this deliverable, if any,
    and how did you resolve them?
\end{enumerate}

\newpage{}

\section*{Appendix --- Team Charter}

\wss{borrows from
  \href{https://engineering.up.edu/industry_partnerships/files/team-charter.pdf}
{University of Portland Team Charter}}

\subsection*{External Goals}

\wss{What are your team's external goals for this project? These are not the
  goals related to the functionality or quality fo the project.  These are the
  goals on what the team wishes to achieve with the project.  Potential goals are
  to win a prize at the Capstone EXPO, or to have something to talk about in
interviews, or to get an A+, etc.}

\subsection*{Attendance}

\subsubsection*{Expectations}

\wss{What are your team's expectations regarding meeting attendance (being on
time, leaving early, missing meetings, etc.)?}

\subsubsection*{Acceptable Excuse}

\wss{What constitutes an acceptable excuse for missing a meeting or a deadline?
What types of excuses will not be considered acceptable?}

\subsubsection*{In Case of Emergency}

\wss{What process will team members follow if they have an emergency and cannot
  attend a team meeting or complete their individual work promised for a team
deliverable?}

\subsection*{Accountability and Teamwork}

\subsubsection*{Quality}

\wss{What are your team's expectations regarding the quality
  of team members' preparation for team meetings and the quality of the
deliverables that members bring to the team?}

\subsubsection*{Attitude}

\wss{What are your team's expectations regarding team members' ideas,
  interactions with the team, cooperation, attitudes, and anything else regarding
  team member contributions?  Do you want to introduce a code of conduct?  Do you
want a conflict resolution plan?  Can adopt existing codes of conduct.}

\subsubsection*{Stay on Track}

\wss{What methods will be used to keep the team on track? How will your team
  ensure that members contribute as expected to the team and that the team
  performs as expected? How will your team reward members who do well and manage
  members whose performance is below expectations?  What are the consequences for
someone not contributing their fair share?}

\wss{You may wish to use the project management metrics collected for the TA and
instructor for this.}

\wss{You can set target metrics for attendance, commits, etc.  What are the
  consequences if someone doesn't hit their targets?  Do they need to bring the
  coffee to the next team meeting?  Does the team need to make an appointment with
their TA, or the instructor?  Are there incentives for reaching targets early?}

\subsubsection*{Team Building}

\wss{How will you build team cohesion (fun time, group rituals, etc.)? }

\subsubsection*{Decision Making}

\wss{How will you make decisions in your group? Consensus?  Vote? How will you
handle disagreements? }

\end{document}
