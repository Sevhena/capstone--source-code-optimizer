\documentclass{article}

\usepackage{booktabs}
\usepackage{tabularx}

\title{Development Plan\\ \progname}

\author{\authname}

\date{}

%% Comments

\usepackage{color}

\newif\ifcomments\commentstrue %displays comments
%\newif\ifcomments\commentsfalse %so that comments do not display

\ifcomments
\newcommand{\authornote}[3]{\textcolor{#1}{[#3 ---#2]}}
\newcommand{\todo}[1]{\textcolor{red}{[TODO: #1]}}
\else
\newcommand{\authornote}[3]{}
\newcommand{\todo}[1]{}
\fi

\newcommand{\wss}[1]{\authornote{blue}{SS}{#1}} 
\newcommand{\plt}[1]{\authornote{magenta}{TPLT}{#1}} %For explanation of the template
\newcommand{\an}[1]{\authornote{cyan}{Author}{#1}}

%% Common Parts

\newcommand{\progname}{Software Engineering} % PUT YOUR PROGRAM NAME HERE
\newcommand{\authname}{Team 6, EcoOptimizers
\\ Nivetha Kuruparan
\\ Sevhena Walker
\\ Tanveer Brar
\\ Mya Hussain
\\ Ayushi Amin} % AUTHOR NAMES                  

\usepackage{hyperref}
    \hypersetup{colorlinks=true, linkcolor=blue, citecolor=blue, filecolor=blue,
                urlcolor=blue, unicode=false}
    \urlstyle{same}
                                


\begin{document}

\maketitle

\begin{table}[hp]
  \caption{Revision History} \label{TblRevisionHistory}
  \begin{tabularx}{\textwidth}{llX}
    \toprule
    \textbf{Date} & \textbf{Developer(s)} & \textbf{Change}\\
    \midrule
    September 18th, 2024 & All & Created first draft of document\\
    September 23rd, 2024 & All & Finalized document\\
    \bottomrule
  \end{tabularx}
\end{table}

\newpage{}

\noindent
This document outlines the development plan for improving the energy efficiency of
engineered software through refactoring. It includes details on intellectual property,
team roles, workflow structure, and project scheduling. Additionally, the plan covers
expected technologies, coding standards, and proof of concept demonstrations, providing
a clear roadmap for the project's progression.

\section{Confidential Information?}

No confidential information to protect.
\section{IP to Protect}

\hspace{\parindent}The software and associated documentation files for this project are protected by copyright, outlined in the document \href{https://github.com/ssm-lab/capstone--source-code-optimizer/blob/documentation/LICENSE}{ here}(referred to as "License"). The License does not grant any party the rights to modify, merge, publish, distribute, sublicense, or sell the software without explicit permission .
Unauthorized use, modification, or distribution of the software is prohibited and may result in legal action.
\\
Permission is granted on a case-by-case basis, non-transferable and non-exclusive. No rights to exploit the software commercially or otherwise are granted without prior written consent from the copyright holders.

\section{Copyright License}

See LICENSE file in root of repository.
\href{https://github.com/ssm-lab/capstone--source-code-optimizer/blob/main/LICENSE}{Click to Open}

\section{Team Meeting Plan}

The team will meet multiple times a week, once during Monday tutorial time and throughout the week as issues or concerns arise. The meetings will be conducted either online through a Teams meeting or in-person on campus. The team
will hold an official meeting with the industry advisor once a week yet the time has not been decided. The meeting with the advisor will be online on Teams for the first 3 weeks, followed by in-person meetings later on. Meetings itself will be strutured as follows:
\begin{enumerate}
  \item Each member will take turns giving a short recap of work they have accomplished throughout the week.
  \item Members will voice any concerns or issues they may be facing.
  \item Team will form a discussion and make decisions for the project.
  \item Any/all questions will be documented for the next meeting with the advisor.
\end{enumerate}
The team will go ahead and use Issues on Github to add anything they may want to talk about in the next meeting as the week progresses in order to have some form of agenda for the next meeting.

\section{Team Communication Plan}

\textbf{Issues}: GitHub \\
\textbf{Meetings}: Microsoft Teams \\
\textbf{Meetings (with advisor)}: Discord \\
\textbf{Informal Project Discussion}: WhatsApp

\section{Team Member Roles}

\hspace{\parindent}As a team, we will all function as developers, sharing responsibilities
for creating issues, coding, testing, and documentation in the early stages. Specific roles
will be defined as the project evolves, allowing for flexibility and collaboration. \\

During our scheduled meetings (with the supervisor, within the team, etc.), we will follow
an Agile Scrum structure, incorporating additional roles such as Scrum Master and Scribe.
These roles will rotate weekly, with the Scrum Master responsible for organizing and leading
the meetings, and the Scribe tasked with documenting key details. This approach ensures active
participation and shared responsibility amongst team members. \\

We have chosen not to designate a team leader, as we all possess similar skills and knowledge.
Instead, we aim to work collaboratively to resolve any challenges that arise.

\section{Workflow Plan}

The repository will contain two main persistent branches (branching off main) throughout the project:
\textbf{dev} and \textbf{documentation}. These branches, that we will henceforth call "epic" branches, will be used for technical
software development and documentation changes respectively. \\

The average workflow for the project will proceed as follows:
\begin{enumerate}
  \item Pull changes from the appropriate \textbf{epic} branch
  \item Create working branch from \textbf{epic} branch with the format [main contributor name]/[descriptive related to topic of changes]
  \item Create sub-working-branch from previous branch if necessary
  \item Commit changes with descriptive name
  \item Create \textbf{unit tests} for said changes
  \item Create pull request to merge changes from sub/working branch into working/epic branch
  \item Wait for all tests run with \textbf{GitHub Actions} to pass
  \item Wait for 2 approvals from teammates other than the one who created the Pull request
  \item Merge changes into target branch (working or epic)
\end{enumerate}

\noindent
\textbf{GitHub Issues} will be used to track and manage bugs, feature requests, and development tasks. Team members can report issues, propose enhancements, and assign tasks to specific individuals. Each issue will be labelled (e.g., bug, enhancement, team-meeting) for easy categorisation, linked to relevant \textbf{milestones}, and tracked through \textbf{GitHub Project} boards. Comments and code references will be exploited to allow for collaboration and discussion on potential solutions. Issues will also be integrated with pull requests, so they’ll automatically close once fixes are merged. \\

\noindent
For situations where a certain type of issue is projected to be created at a steady frequency, templates will be created to facilitate their creation. \\

\noindent
\textbf{Milestones} will be used to organize commits and pull requests for major deliverables. Once all working branches associated to the milestone have been merged into the appropriate epic branch, the team will go through the relevant prepared checklist to make sure that all requirements have been met. Once this is done, the epic branch will be merged into main as one pull request along with the checklist. \\

\noindent
The \textbf{CI/CD} pipeline will be implemented via GitHub to improve automated testing as well as facilitate the feedback loop for the machine learning models used by the library. More detailed information will be added later. \\

\section{Project Decomposition and Scheduling}

\hspace{\parindent} The project is hosted on GitHub under the organization of Sustainable Systems and Methods (SSM) Lab and can be accessed at this \href{https://github.com/ssm-lab/capstone--source-code-optimizer}{link}.

The team currently has one GitHub Project setup to serve as a visual tool not only to track deadlines for all tasks but also for accountability  for assigned tasks. The project can include cards for:
\begin{itemize}
  \item Issues for course related deliverables
  \item Lecture and Meeting logs
  \item Issues for project specific tasks(such as building a certain component).
\end{itemize}

\wss{How will the project be scheduled?  This is the big picture schedule, not
  details. You will need to reproduce information that is in the course outline
for deadlines.}

\section{Proof of Concept Demonstration Plan}

For context, our POC will consist of roughly the following steps:
\begin{enumerate}
  \item Determine the code smells we want to address for energy saving.
  \begin{itemize}
    \item These are items like but not limited to: Large Class (LC), Long Parameter List (LPL), Long Method (LM), Long Message Chain (LMC), Long Scope Chaining (LSC), Long Base Class List (LBCL), Useless Exception Handling (UEH), Long Lambda Function (LLF), Complex List Comprehension (CLC), Long Element Chain (LEC), and Long Ternary Conditional Expression (LTCE).     
  \end{itemize} 
  \item Determine the detectability of a specific code smell 
  \begin{itemize}
    \item Many of these code smells are detectable using linters like Pylint, Flake8 and bandit. Hence the detection technology already exists if we choose to use it, and or the tools have been made before if we intend to remake them, proving they are possible to construct.
  \end{itemize} 
  \item Determine the appropriate refactorings for a particular detected smell that results in decreased energy consumption.
  \begin{itemize}
    \item There are many tools such as Pyjoule that we can use to measure the energy consumed by a piece of code. This step will involve various phases of trial and error as it is not a 1-1 trivial solution. There could be various refactorings possible for a given situation that all result in different energy consumption levels. We want our tool to choose the most optimal refactoring possible. For our POC this can exist as an algorithm. For our final project we can attempt to implement a neural network to choose between refactorings. There are also prebuilt free to use libraries we can implement to perform simple refactorings.
  \end{itemize} 
  \item Once we determine preset algorithms mapping detected smells to their appropriate refactorings, we want to then make those changes in the code, measure the energy consumption and test it against the original code ensuring it is less.
  \item The code must then ensure that the original code functionallity is preserved. If it is not a different refactoring is required. 
  \begin{itemize}
    \item This can be done by testing the original test suite for the code against the new one. This original test suite can be a required argument for the user.
  \end{itemize} 
\end{enumerate}


The following is a list of primary risks and how potential results from the POC could mitigate them.
\begin{enumerate}

  \item The refactorings we propose may not reduce the code's energy consumption.
  \begin{itemize}
    \item If this occurs the team will have to re-analyze the results, revisit refactoring strategies and conduct additional testing to ensure that the proper refactorings are occuring. Code exists in two forms, fully optimized, and not fully optimized. If it is the later, using a new strategy should work provided our team has the expertise to find the correct refactoring. If it is the former the correct use case of our tool is simply the result that no new optimal refactorings could be found. This is a valid result for some use cases.
  \end{itemize} 

  \item Energy is decreased but functionality of the code is modified. 
  \begin{itemize}
    \item If this occurs then the refactoring we applied was incorrect for the given code. A new refactoring should occur. Finding valid refactorings can be implemented iteratively, it is okay for the code to get a couple wrong so long as the final answer is correct. Requiring the user to submit a test suite for the original code can ensure that the code is not modified beyond its purpose. We can also add error handling that lets the user be in charge of the refactorings by making them suggestions instead of absolutes, that allows for the software engineer to have more control over what their final code looks like.
  \end{itemize} 

\end{enumerate}

Other smaller risks for our tool include:
\begin{itemize}
  \item Integration challenges: Challenges adding our tool to people's CI/CD pipelines. What should that look like? How can we make it as accessible as possible? 
  \begin{itemize}
    \item Potential answer: If we make a plugin we should try and make it compatible with a widely used IDE.
  \end{itemize} 
  \item Software developers not wanting to adopt energy saving into their code bases. How can we make it as user friendly as possible?
  \begin{itemize}
    \item Potential answer: Ease of use and the success of the tool will be the primary factors in it's use. For corectness we can ensure we have thorough testing to ensure the tool works as promised. For user experiences we can conduct small trials or focus groups seeing what people prefer.  
  \end{itemize} 
  \item Code performance issues: A lot of code choices are tradeoffs. Are there situations in which we would be decreasing runtime or performance to achieve a better energy output? If so what consequences does this have?
  \begin{itemize}
    \item Potential answer: Allowing the tool to act as a suggestion rather than a hard refactor and giving the developer the choice to undo changes could help the engineer have more control over the specifications that matter to them the most. 
  \end{itemize} 
\end{itemize} 

\section{Expected Technology}

\wss{What programming language or languages do you expect to use?  What external
  libraries?  What frameworks?  What technologies.  Are there major components of
  the implementation that you expect you will implement, despite the existence of
  libraries that provide the required functionality.  For projects with machine
learning, will you use pre-trained models, or be training your own model?  }

\wss{The implementation decisions can, and likely will, change over the course
  of the project.  The initial documentation should be written in an abstract way;
  it should be agnostic of the implementation choices, unless the implementation
  choices are project constraints.  However, recording our initial thoughts on
  implementation helps understand the challenge level and feasibility of a
  project.  It may also help with early identification of areas where project
members will need to augment their training.}

Topics to discuss include the following:

\begin{itemize}
  \item Specific programming language
  \item Specific libraries
  \item Pre-trained models
  \item Specific linter tool (if appropriate)
  \item Specific unit testing framework
  \item Investigation of code coverage measuring tools
  \item Specific plans for Continuous Integration (CI), or an explanation that CI
    is not being done
  \item Specific performance measuring tools (like Valgrind), if
    appropriate
  \item Tools you will likely be using?
\end{itemize}

\wss{git, GitHub and GitHub projects should be part of your technology.}

\section{Coding Standard}

\wss{What coding standard will you adopt?}

\newpage{}

\section*{Appendix --- Reflection}

\wss{Not required for CAS 741}

The purpose of reflection questions is to give you a chance to assess your own
learning and that of your group as a whole, and to find ways to improve in the
future. Reflection is an important part of the learning process.  Reflection is
also an essential component of a successful software development process.  

Reflections are most interesting and useful when they're honest, even if the
stories they tell are imperfect. You will be marked based on your depth of
thought and analysis, and not based on the content of the reflections
themselves. Thus, for full marks we encourage you to answer openly and honestly
and to avoid simply writing ``what you think the evaluator wants to hear.''

Please answer the following questions.  Some questions can be answered on the
team level, but where appropriate, each team member should write their own
response:


\subsubsection*{Mya Hussain Reflection}

\begin{enumerate}
  \item \textit{Why is it important to create a development plan prior to starting the project?}

    Development plans act as good starting points by establishing clear objectives, recourse requirements, timelines and by 
    assigning accountability within a project. It improves overall communication and understanding between members by solidifying 
    the intent and requirements of a project. From my experience in industry, developers who don’t have a good grasp of what they’re 
    developing and why they’re developing it typically deliver products that either fail to meet the primary goal or prove useless to 
    the user. In real life, you develop products for people with various backgrounds. People of different disciplines speak in different 
    technical tongues. Development plans are critical in syncing many people from various disciplines towards a common goal. They can also 
    further can divide the roles and responsibilities between the developers and later aid in making development decisions that push 
    progress towards the primary goals defined in the document.
    
  \item \textit{What disagreements did your group have in this deliverable, if any, and how did you resolve them?}

    I would be lying if I were to say that my team had disagreements during the first week of working on the project. The easy answer to 
    this question is to make up a disagreement of low significance and say we worked through it using open democratic discussion and 
    compromise. The truth is that it is too early in the project for us to be disagreeing with each other. We’re in our “honeymoon stage” 
    where all members are excited to work on the project and are optimistic about the possible results we could achieve. Let us be in love 
    during week one. We have many more weeks left to disagree.
    
\end{enumerate}

\subsubsection*{Ayushi Amin Reflection}

\begin{enumerate}
  \item \textit{Why is it important to create a development plan prior to starting the
    project?}
    
    A developement plan is important because it serves as a clear roadmap for the project and ensures that all 
    team members understand the the project scope, objectives and timeline. A clear plan helps to identify potential risks
    and areas of concern/challenges that can be caught at an early stage so a solution can be crafted early on. It also helps 
    in resource allocation by ensuring that the necessary tools and budget are available when needed.
    
  \item \textit{What disagreements did your group have in this deliverable, if any, and how did you resolve them?}
  
    To be quite frank, our team did not have any disagreements during the first couple weeks of working together.
    
\end{enumerate}

\subsubsection*{Sevhena Walker Reflection}

\begin{enumerate}

  \item \textit{Why is it important to create a development plan prior to starting the project?}

    When starting a project with a large scope, it is easy to get lost in all the features you want to implement in the project. 
    By establish an action plan from the start, you allow yourself and your team to develop concrete goals and needs for your system. 
    It is necessary to take a step back and analyze how exactly the components of your system are expected to interact with each other 
    and what the end result is suppose to be.\\

    Without this planning stage, you are essentially going in blind with only a vague understanding and what you need to do. There is 
    no way to organize task between team members efficiently either since specific components are sparsely detailed or non-existent. The 
    project will be in a constant state of "debugging" you could say, constantly trying to figure out how this feature will work with the 
    next and the next and so on. It would be like trying to build a house while only ever looking at the next couple meters in front of you.

  \item \textit{What disagreements did your group have in this deliverable, if any, and how did you resolve them?}
  I honestly cannot say that we had any disagreements as yet. Our team tried our best to discuss what our goals for the project were and 
  everything was well communicated so that everyone was on the same page. We are at the beginning stages of a project that will implement 
  technology is mostly new to us. We are excited, but also somewhat ignorant to some details which makes it hard to "disagree".

\end{enumerate}

\subsubsection*{Tanveer Brar}

\begin{enumerate}
  \item \textit{Why is it important to create a development plan prior to starting the project?}

   After having created one, I believe a development plan is important as it can be useful for the project in multiple ways. For one, it is a written record of the team's expected practices related to roles, communication, etc.(which ensures that everyone is on the same page with these practices). It can also be used by instructors as the source of truth if there are any conflicts in the project team.\\

   Secondly, with the pre-written format/prompts we were compelled to think about some logistics that we would otherwise have ignored(such as communication plan and coding standards). In short, it set a great starting note for the project so we were aware of our assumptions and were also compelled to think about less obvious aspects of project planning.

  \item \textit{What disagreements did your group have in this deliverable, if any, and how did you resolve them?}

  We haven’t had any disagreements in our group for this deliverable. There were multiple instances where each of us had different ideas on how to approach a particular issue. After weighing in the pros and cons of each idea together, we ended up picking one of them unanimously. It was possible since we value each other’s thought process and take a non partial approach when deliberating.

\end{enumerate}

\subsubsection*{Group Answers}

\textit{In your opinion, what are the advantages and disadvantages of using CI/CD?}
  
    CI/CD or Continuous Integration/Continuous Deployment is a two-step process where continuous integration means automatically 
    integrating code changes from multiple contributors into a shared repository for every commit, and continuous deployment involves 
    keeping your code base deployable at any time.
    Key practices in CI/CD include automated testing and frequent commits. This leads to better code quality as bugs and conflicts between 
    users are caught early on and easily rolled back or fixed near their date of creation. Another advantage is a faster time to market as 
    code is integrated and deployed on development allowing teams to deliver features/fixes to users quicker. It also allows for improved 
    collaboration through feedback as features pushed to shared repositories can be easily reviewed by the team.
    Disadvantages include the initial setup complexity of the CI/CD pipeline as this can be time consuming and a learning curve to those 
    new to the process. It also requires a lot of maintenance and overhead to continuously commit, push, pull, test and merge code diverging
    from time that could be spent developing/improving a feature.

    In CI/CD, there is a constant cycle of commiting code to a code base, and running automated tests on that piece of code before 
    integrating it into the system. This constant and frequent cycle helps detect problems early leading toa more robust system. 
    Furthermore, stress is taken away from the programmer when it comes to building and deploying software as it is all automated.\\
    The flip side to this is, of course, that the developer must initially set up the CI/CD environment. All the automated tests must 
    first be written, and they must be written \textit{well}. The same goes for the build and deployment workflows. This setup can be 
    complex and lengthy and most of all, it requires the developer to know how to do it. Furthermore, it must be \textit{maintained}. 
    Other risks include a serious over-reliance in automation which could potentially introduce security flaws that weren't thought of 
    before.
  
    In other terms:
    Advantages:
    \begin{enumerate}
      \item Faster development since changes would be pushed constantly as testing for new/fixed features would be successful
      \item Shorter delivery times
      \item Improved code quality (automated testing)
      \item Easier change monitoring and rollback since there are 5 members and multiple branches and changes based on the member
      \item Increased efficiency and productivity
    \end{enumerate}
     
    Disadvantages
    \begin{enumerate}
      \item Requires strong discipline and commitment from the entire team and may be a learning curve for the team
      \item May be challenging to implement for small teams since there is only 5 members
      \item Initial setup and configuration can be complex and time-consuming since the team will be setting this up from scratch
    \end{enumerate}

\newpage{}

\section*{Appendix --- Team Charter}

\wss{borrows from
  \href{https://engineering.up.edu/industry_partnerships/files/team-charter.pdf}
{University of Portland Team Charter}}

\subsection*{External Goals}

\wss{What are your team's external goals for this project? These are not the
  goals related to the functionality or quality fo the project.  These are the
  goals on what the team wishes to achieve with the project.  Potential goals are
  to win a prize at the Capstone EXPO, or to have something to talk about in
interviews, or to get an A+, etc.}

\subsection*{Attendance}

\subsubsection*{Expectations}

\wss{What are your team's expectations regarding meeting attendance (being on
time, leaving early, missing meetings, etc.)?}

\subsubsection*{Acceptable Excuse}

\wss{What constitutes an acceptable excuse for missing a meeting or a deadline?
What types of excuses will not be considered acceptable?}

\subsubsection*{In Case of Emergency}

\wss{What process will team members follow if they have an emergency and cannot
  attend a team meeting or complete their individual work promised for a team
deliverable?}

\subsection*{Accountability and Teamwork}

\subsubsection*{Quality}

Our team has the following expectations regarding the quality of preparation for meetings and the deliverables brought to the team:

\begin{itemize}
    \item \textbf{Meeting Preparation}:
    \begin{itemize}
        \item Team members are expected to arrive at meetings fully prepared, having reviewed relevant materials and completed their assigned tasks in advance.
        \item Each member should come ready to discuss their progress, share insights, and address any challenges they are facing.
        \item Members should ensure that their updates are clear and concise, allowing meetings to stay focused and productive.
    \end{itemize}

    \item \textbf{Deliverables Quality}:
    \begin{itemize}
        \item All deliverables must meet the team’s agreed-upon standards, demonstrating a high level of accuracy, thoroughness, and attention to detail.
        \item Each deliverable should be carefully reviewed by each member before submission to avoid any errors or incomplete work.
        \item Deliverables must align with the project's requirements and deadlines, ensuring they are both functional and meet the expected quality criteria.
    \end{itemize}
    
    \item \textbf{Accountability and Feedback}:
    \begin{itemize}
        \item Team members are responsible for completing their work to a high standard, communicating any issues early if they need assistance or more time.
        \item Feedback on deliverables should be welcomed by all members, and revisions should be made promptly to improve the overall quality of the team’s output.
    \end{itemize}
    
\end{itemize}

By maintaining these expectations, our team will ensure that meetings are efficient and that all deliverables reflect a professional and high-quality standard.

\subsubsection*{Attitude}

Our team has established the following \textbf{expectations} for team members' contributions, interactions, and cooperation to ensure a productive and respectful working environment:

\begin{itemize}
    \item \textbf{Respectful Communication}: All team members are expected to listen to each other’s ideas and provide constructive feedback. Communication should remain respectful, even in cases of disagreement.
    \item \textbf{Open Collaboration}: Each member is encouraged to share their ideas openly. Everyone should be willing to collaborate and help each other achieve team goals.
    \item \textbf{Accountability}: Team members are responsible for completing their tasks by the agreed-upon deadlines. If a member is struggling, they are expected to ask for help or communicate early.
    \item \textbf{Positive Attitude}: Maintaining a positive attitude, especially in challenging moments, is essential for team morale. Each member should encourage and support their teammates.
    \item \textbf{Commitment to Quality}: Every team member is expected to contribute to the project with their best effort, ensuring that the final product reflects high standards of quality.
\end{itemize}

We adopt the following \textbf{code of conduct} to guide behavior and interaction among team members:

\begin{itemize}
    \item \textbf{Inclusivity}: Our team values diversity and is committed to creating an inclusive environment where everyone feels welcome and valued, regardless of background, experience, or opinion.
    \item \textbf{Professionalism}: Members will engage professionally, refraining from any inappropriate or offensive language or behavior. This applies to both in-person and online interactions.
    \item \textbf{Collaboration and Feedback}: We encourage constructive feedback and expect team members to accept and provide feedback in a way that helps everyone grow. Criticism should be focused on the work, not the individual.
    \item \textbf{No Tolerance for Harassment}: Harassment of any kind will not be tolerated. Any issues will be reported immediately and addressed in a structured manner.
\end{itemize}

To manage conflicts or disagreements that may arise during the project, we have a \textbf{conflict resolution plan} in place:

\begin{enumerate}
    \item \textbf{Address the Issue Directly}: If a conflict arises, the involved members should first try to resolve the issue directly through a respectful discussion.
    \item \textbf{Mediation by a Neutral Member}: If the conflict cannot be resolved, the team will appoint a neutral team member to act as a mediator to facilitate a discussion and find common ground.
    \item \textbf{Escalation to Instructor/TA}: In the event that the conflict cannot be resolved within the team, the issue will be escalated to the instructor or TA for further guidance and resolution.
    \item \textbf{Follow-Up and Monitoring}: After resolving the conflict, the team will continue to monitor the situation to ensure that the issue does not resurface and that team dynamics remain positive.
\end{enumerate}

By adhering to these expectations, the code of conduct, and our conflict resolution plan, we aim to maintain a positive, collaborative, and respectful team environment.

\subsubsection*{Stay on Track}

To keep our team on track, we will implement the following methods:

\begin{enumerate}
    \item \textbf{Regular Check-ins and Progress Updates}: We will hold \textit{weekly meetings} where each member will 
    provide an update on their tasks and progress and any concerns or troubles they faced. These updates will help us identify issues 
    early and adjust accordingly to stay on schedule.
    \item \textbf{Performance Metrics}: We will track the following key metrics:
    \begin{itemize}
        \item \textit{Attendance} at meetings and check-ins will be documented through Issues on GitHub.
        \item \textit{Commits to the repository}, ensuring steady contributions.
        \item \textit{Task completion rates}, ensuring deadlines are met.
    \end{itemize}
    \item \textbf{Rewards for High Performers}: To encourage good performance, we will recognize and celebrate team members who meet or 
    exceed expectations. Informal rewards may include public recognition during meetings or assigning leadership roles in future tasks.
    \item \textbf{Managing Under performance}: If a team member's performance is below expectations:
    \begin{itemize}
        \item We will start with a \textit{team conversation} to understand any obstacles and offer support.
        \item If under performance continues, consequences may include \textit{more tasks} for milestone or in severe cases, a meeting with
        the TA or instructor.
    \end{itemize}
    \item \textbf{Consequences for Not Contributing}: If a team member does not contribute their fair share:
    \begin{itemize}
        \item They may be assigned additional \textit{tasks} to balance the workload.
        \item In serious cases, the issue will be brought up to the TA or instructor.
    \end{itemize}
    \item \textbf{Incentives for Meeting Targets Early}: Members who consistently meet or exceed their targets will be rewarded with more 
    desirable tasks as per their wants, such as leadership roles in key project components, helping to build their leadership experience. They
    will get first pick on tasks for the next team milestone.

\end{enumerate}

\subsubsection*{Team Building}

For team building events, the team has decided to have bi-weekly hangouts to bond and build relationships. The hangouts can 
attending on-campus events together, getting food or bubble tea on/off campus and more.

\subsubsection*{Decision Making}

In our group, our primary way of making decisions will be through consensus. We believe that it is important
to include everyone in the decision-making process so it can lead to better outcomes and strong group work.
In certain situations where consensus cannot be reached, the group will take a vote and each member will have
equal say and the decision will be based on the majority rule. We will make sure all group members had a chance
to voice their opinions before making the final decision through consensus or a vote.

\vspace{10pt}
\textit{To Handle disagreements: The team will address each disagreement directly and respectfully.}

\begin{enumerate}
  \item Allow all team members to express their concerns and opinions without interruption, ensuring everyone 
  feels heard.
  \item Keep the focus of the discussion on the topic at hand rather than personal feelings.
  \item When necessary, we may appoint a neutral party to facilitate the discussion and help guide it to a resolution.
  \item If a resolution is not found or the disagreement persists after the resolution is found, we will aim to 
  revisit our project goals and objectives to ensure that our decisions align with our common purpose.
\end{enumerate}
By following these strategies, we aim to maintain a collaborative and positive team environment while effectively 
managing decisions and conflicts.

\end{document}
