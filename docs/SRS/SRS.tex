% THIS DOCUMENT IS FOLLOWS THE VOLERE TEMPLATE BY Suzanne Robertson and James Robertson
% ONLY THE SECTION HEADINGS ARE PROVIDED
%
% Initial draft from https://github.com/Dieblich/volere
%
% Risks are removed because they are covered by the Hazard Analysis
\documentclass[12pt]{article}

\usepackage[round]{natbib}
\usepackage[letterpaper, portrait, margin=1in]{geometry}
\usepackage{booktabs}
\usepackage{siunitx}
\usepackage{tabularx}
\usepackage{enumerate}
\usepackage{hyperref}
\hypersetup{
    bookmarks=true,         % show bookmarks bar?
      colorlinks=true,      % false: boxed links; true: colored links
    linkcolor=red,          % color of internal links (change box color with linkbordercolor)
    citecolor=green,        % color of links to bibliography
    filecolor=magenta,      % color of file links
    urlcolor=cyan           % color of external links
}

\newcommand{\lips}{\textit{Insert your content here.}}

%% Comments

\usepackage{color}

\newif\ifcomments\commentstrue %displays comments
%\newif\ifcomments\commentsfalse %so that comments do not display

\ifcomments
\newcommand{\authornote}[3]{\textcolor{#1}{[#3 ---#2]}}
\newcommand{\todo}[1]{\textcolor{red}{[TODO: #1]}}
\else
\newcommand{\authornote}[3]{}
\newcommand{\todo}[1]{}
\fi

\newcommand{\wss}[1]{\authornote{blue}{SS}{#1}} 
\newcommand{\plt}[1]{\authornote{magenta}{TPLT}{#1}} %For explanation of the template
\newcommand{\an}[1]{\authornote{cyan}{Author}{#1}}

%% Common Parts

\newcommand{\progname}{Software Engineering} % PUT YOUR PROGRAM NAME HERE
\newcommand{\authname}{Team 6, EcoOptimizers
\\ Nivetha Kuruparan
\\ Sevhena Walker
\\ Tanveer Brar
\\ Mya Hussain
\\ Ayushi Amin} % AUTHOR NAMES                  

\usepackage{hyperref}
    \hypersetup{colorlinks=true, linkcolor=blue, citecolor=blue, filecolor=blue,
                urlcolor=blue, unicode=false}
    \urlstyle{same}
                                


\begin{document}

\title{Software Requirements Specification for \progname: subtitle describing software} 
\author{\authname}
\date{\today}
	
\maketitle

~\newpage

\pagenumbering{roman}

\tableofcontents

~\newpage

\section*{Revision History}

\begin{tabularx}{\textwidth}{p{3cm}p{2cm}X}
\toprule {\textbf{Date}} & {\textbf{Version}} & {\textbf{Notes}}\\
\midrule
Date 1 & 1.0 & Notes\\
Date 2 & 1.1 & Notes\\
\bottomrule
\end{tabularx}

~\\

~\newpage
\section{Purpose of the Project}
\subsection{User Business}
\lips
\subsection{Goals of the Project}
\lips
\section{Stakeholders}
\subsection{Client}
\lips
\subsection{Customer}
\lips
\subsection{Other Stakeholders}
\lips
\subsection{Hands-On Users of the Project}
\lips
\subsection{Personas}
\lips
\subsection{Priorities Assigned to Users}
\lips
\subsection{User Participation}
\lips
\subsection{Maintenance Users and Service Technicians}
\lips

\section{Mandated Constraints}
\subsection{Solution Constraints}
\lips
\subsection{Implementation Environment of the Current System}
\lips
\subsection{Partner or Collaborative Applications}
\lips
\subsection{Off-the-Shelf Software}
\lips
\subsection{Anticipated Workplace Environment}
\lips
\subsection{Schedule Constraints}
\lips
\subsection{Budget Constraints}
\lips
\subsection{Enterprise Constraints}
\lips

\section{Naming Conventions and Terminology}
\subsection{Glossary of All Terms, Including Acronyms, Used by Stakeholders
involved in the Project}
\lips

\section{Relevant Facts And Assumptions}
\subsection{Relevant Facts}
\lips
\subsection{Business Rules}
\lips
\subsection{Assumptions}
\lips

\section{The Scope of the Work}
\subsection{The Current Situation}
\lips
\subsection{The Context of the Work}
\lips
\subsection{Work Partitioning}
\lips
\subsection{Specifying a Business Use Case (BUC)}
\lips

\section{Business Data Model and Data Dictionary}
\subsection{Business Data Model}
\lips
\subsection{Data Dictionary}
\lips

\section{The Scope of the Product}
\subsection{Product Boundary}
\lips
\subsection{Product Use Case Table}
\lips
\subsection{Individual Product Use Cases (PUC's)}
\lips

\section{Functional Requirements}
\subsection{Functional Requirements}
\lips

\section{Look and Feel Requirements}
\subsection{Appearance Requirements}
\lips
\subsection{Style Requirements}
\lips

\section{Usability and Humanity Requirements}
\subsection{Ease of Use Requirements}
\lips
\subsection{Personalization and Internationalization Requirements}
\lips
\subsection{Learning Requirements}
\lips
\subsection{Understandability and Politeness Requirements}
\lips
\subsection{Accessibility Requirements}
\lips

\section{Performance Requirements}
\subsection{Speed and Latency Requirements}
\lips
\subsection{Safety-Critical Requirements}
\lips
\subsection{Precision or Accuracy Requirements}
\lips
\subsection{Robustness or Fault-Tolerance Requirements}
\lips
\subsection{Capacity Requirements}
\lips
\subsection{Scalability or Extensibility Requirements}
\lips
\subsection{Longevity Requirements}
\lips

\section{Operational and Environmental Requirements}
% \begin{enumerate}[{OER-WE}1. ]
	% \item \emph{}\\
  %   {\bf Rationale:} \\
  %   {\bf Fit Criterion:}  
% \end{enumerate}
\subsection{Expected Physical Environment}

\begin{enumerate}[{OER-EP}1. ]
	\item \emph{The product shall be used in temperatures ranging from \SI{10}{\celsius} - \SI{35}{\celsius}.}\\
    {\bf Rationale:} A computer's safe operating range is \SI{10}{\celsius} - \SI{35}{\celsius} ~\citep{PCTemp}. If the computer doesn't work then it is not possible to use the refactoring library. \\
    {\bf Fit Criterion:} The computer turns on, and no temperature warning is issued.
  \item \emph{The product shall be used in proximity to a stable power supply.}\\
  {\bf Rationale:} As a coding library, the product depends on the continuing operation of the computer system it is used on. Should the computer lose power, the refactoring library will see its processes halted. \\
  {\bf Fit Criterion:} The computer is connected to a power outlet or the computer possesses charge on its battery. 
\end{enumerate}

\subsection{Wider Environment Requirements}
\begin{enumerate}[{OER-WE}1. ]
	\item \emph{The system must align with widely used emissions standards (e.g., GRI 305, GHG, ISO 14064) ~\citep{GHG,ISO14064,GRI305}.}\\
    {\bf Rationale:} Providing metrics tailored to these standards, makes the library reporting tool more attractive to users part of companies looking to reduce their ecological footprint. \\
    {\bf Fit Criterion:} The emissions tracked by the standards are present in the reported metrics.  
\end{enumerate}

\subsection{Requirements for Interfacing with Adjacent Systems}
\begin{enumerate}[{OER-IAS}1. ]
	\item \emph{The refactoring library must provide integration capabilities with GitHub Actions.}\\
    {\bf Rationale:} This will allow the automation of the refactoring process within existing workflows to ensure that energy-efficient practices are consistently applied during continuous integration.\\
    {\bf Fit Criterion:} The library is available to use via GitHub Actions when writing workflows.
  \item \emph{The library should be compatible with the Visual Studio Code (VSCode) IDE.}\\
    {\bf Rationale:} Developers will be able to refactor code easily without leaving their working environment, therefore enhancing the accessibility and usability of the library.\\
    {\bf Fit Criterion:} An extension is available for installation in VSCode marketplace.
  \item \emph{The library should support importing existing codebases and exporting refactored code and energy savings reports in standard formats (e.g., JSON, XML)}\\
    {\bf Rationale:} This ensures that users can easily integrate the library into their existing workflows without significant disruption.\\
    {\bf Fit Criterion:} Developers are able to refactor existing codebases and view relevant metrics.
\end{enumerate}

\subsection{Productization Requirements}
\begin{enumerate}[{OER-PR}1. ]
	\item \emph{The library shall be package with PIP and made available to python users through the public package manager.}\\
    {\bf Rationale:} As a widely used package manager, PIP will be able to distribute the library to any users that wish to use it.\\
    {\bf Fit Criterion:} Users are able to install the library using \texttt{pip install}. 
\end{enumerate}

\subsection{Release Requirements}
\begin{enumerate}[{OER-RL}1. ]
	\item \emph{All core functionalities specified in the requirements must be implemented and tested, including energy consumption measurement, automated refactoring, and reporting features.}\\
    {\bf Rationale:} This will ensures that the library delivers the promised capabilities to users.\\
    {\bf Fit Criterion:} Follows the steps outlined in the Verification and Validation (V\&V) plan.  
  \item \emph{The library must be ready for release by March 17th, 2025.}\\
    {\bf Rationale:} The library must be ready for final demonstration as a requirement of the McMaster University SFRWENG 4G06 Capstone course.\\
    {\bf Fit Criterion:} The project is ready for the final demonstration of the appointed date.
\end{enumerate}

\section{Maintainability and Support Requirements}
\subsection{Maintenance Requirements}
\lips
\subsection{Supportability Requirements}
\lips
\subsection{Adaptability Requirements}
\lips

\section{Security Requirements}
\subsection{Access Requirements}
\lips
\subsection{Integrity Requirements}
\lips
\subsection{Privacy Requirements}
\lips
\subsection{Audit Requirements}
\lips
\subsection{Immunity Requirements}
\lips

\section{Cultural Requirements}
\subsection{Cultural Requirements}
\lips

\section{Compliance Requirements}
\subsection{Legal Requirements}
\lips
\subsection{Standards Compliance Requirements}
\lips

\section{Open Issues}
\lips

\section{Off-the-Shelf Solutions}
\subsection{Ready-Made Products}
\lips
\subsection{Reusable Components}
\lips
\subsection{Products That Can Be Copied}
\lips

\section{New Problems}
\subsection{Effects on the Current Environment}
\lips
\subsection{Effects on the Installed Systems}
\lips
\subsection{Potential User Problems}
\lips
\subsection{Limitations in the Anticipated Implementation Environment That May
Inhibit the New Product}
\lips
\subsection{Follow-Up Problems}
\lips

\section{Tasks}
\subsection{Project Planning}
\lips
\subsection{Planning of the Development Phases}
\lips

\section{Migration to the New Product}
\subsection{Requirements for Migration to the New Product}
\lips
\subsection{Data That Has to be Modified or Translated for the New System}
\lips

\section{Costs}
\lips
\section{User Documentation and Training}
\subsection{User Documentation Requirements}
\lips
\subsection{Training Requirements}
\lips

\section{Waiting Room}
\lips

\section{Ideas for Solution}
\lips

\newpage{}
\section*{Appendix --- Reflection}

The information in this section will be used to evaluate the team members on the
graduate attribute of Lifelong Learning.  Please answer the following questions:

\begin{enumerate}
  \item What knowledge and skills will the team collectively need to acquire to
  successfully complete this capstone project?  Examples of possible knowledge
  to acquire include domain specific knowledge from the domain of your
  application, or software engineering knowledge, mechatronics knowledge or
  computer science knowledge.  Skills may be related to technology, or writing,
  or presentation, or team management, etc.  You should look to identify at
  least one item for each team member.
  \item For each of the knowledge areas and skills identified in the previous
  question, what are at least two approaches to acquiring the knowledge or
  mastering the skill?  Of the identified approaches, which will each team
  member pursue, and why did they make this choice?
\end{enumerate}

\bibliographystyle {plainnat}
\bibliography{../../refs/References}

\end{document}